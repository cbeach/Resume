%LaTeX resume using res.cls
\documentclass[margin]{res}
%\usepackage{helvetica} % uses helvetica postscript font (download helvetica.sty)
%\usepackage{newcent}   % uses new century schoolbook postscript font 
\setlength{\textwidth}{5.1in} % set width of text portion

\usepackage{multicol}
\usepackage{latexsym}

\begin{document}
\pagestyle{empty}
\topmargin 0pt
\leftmargin -.5in

% Center the name over the entire width of resumef:
 \moveleft.5\hoffset\centerline{\large\bf Casey M. Beach}
% Draw a horizontal line the whole width of resume:
 \moveleft\hoffset\vbox{\hrule width\resumewidth height 1pt}\smallskip
% address begins here
% Again, the address lines must be centered over entire width of resume:
 \moveleft.5\hoffset\centerline{3263 SE 178th Avenue}
 \moveleft.5\hoffset\centerline{Portland, OR 97236}
 \moveleft.5\hoffset\centerline{(541) 990-8978}
 \moveleft.5\hoffset\centerline{beachc@gmail.com}
 \moveleft.5\hoffset\centerline{github.com/cbeach}
\begin{resume}

\section{EDUCATION} {\sl Current Student,} Computer Science \\
                Portland State University, Portland, OR. \\
                Expected Graduation Date: March 2012 \\
                {\sl Bachcelor of Science,} Business Administration \\
                Oregon State University, Corvallis, OR, 
 				Graduation Date: March 2007 \\
 				Concentration: Entrepreneurship
 
\section{COMPUTER SKILLS}{\sl The following list includes many of the tools and software packages that I have used}
									\begin{multicols}{3}[\columnsep 2pt \linewidth 400pt]
									{\sl \textbf{Languages \\ Libraries \\ and API's}} \\
									\begin{list}{\labelitemi}{\leftmargin=1em} \itemsep -2pt
										\item C/C++ 
										\item Python
										\item Java 
										\item Octave 
										\item Qt 
										\item Open GL
										\item Latex
									\end{list}
									\vfill
									\columnbreak
									{\sl \textbf{Tools}} 
									\linebreak
									\linebreak
									\linebreak
							 		\begin{list}{\labelitemi}{\leftmargin=1em}\itemsep -2pt
										\item Vim 
										\item GDB
										\item Valgrind 
										\item KCacheGrind
										\item Eclipse
										\item X Code 
										\item git 
										\item Open CV 
										\item Eagle 
										\item AVR-GCC
									\end{list} \itemsep -2pt
									\vfill
									\columnbreak
			                		{\sl \textbf{Techniques}} \\ 								
									\linebreak
									\linebreak
									\begin{list} {\labelitemi}{\leftmargin=1em}\itemsep -2pt
										\item 3D Graphics 
										\item Neural Networks 
										\item Support Vector Machines 
										\item Logistic Regression 
										\item Bayesian Classifiers 
										\item Search Tree Optimizations 
										\item Basic firmware design 
									\end{list}
								\end{multicols}

\section{EXPERIENCE}
				{\sl \textbf{Computer Science Tutor}} \hfill \textbf{2011-Present} \\
                College of Computer Science, Portland State University \hfill
				
				Mentor freshmen, sophomore, and junior students, teaching them basic and intermediate 
				computer science principles and techniques.	I've hosted several introductory workshops 
				on using Linux, and basic Python.
				
				{\sl \textbf{3D Printing and Advanced Fabrication}} \hfill \textbf{2010-2012} \\
                Open Tech-Lab, Portland State University
				
				I have completed an open source RepRap 3D printer. The printer started as a solo project, 
				but since the completion of the initial build a team has been found to assist with further 
				upgrades and research.  I built the RepRap from scratch.  I fabricated the structure, soldered  
				the electronics, and flashed the firmware. 	The goals of this project are to advance the 
				fields of robotic manufacturing and replication, to assist other projects at PSU, and to 
				provide a teaching platform for students to learn more about rapid prototyping. 
				More information can be found at http://projects.cecs.pdx.edu/projects/opentech-reprap. 
 				
\end{resume}
\end{document}




