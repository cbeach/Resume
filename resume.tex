%LaTeX resume using res.cls
\documentclass[overlapped]{res}

%\usepackage{helvetica} % uses helvetica postscript font (download helvetica.sty)
%\usepackage{newcent}   % uses new century schoolbook postscript font 

\usepackage{multicol}
\usepackage{latexsym}
\usepackage{enumitem}
\setlist[1]{noitemsep}

\begin{document}
\pagestyle{empty}

\leavevmode\hbox to \textwidth{\hfil\namefont Casey M. Beach\hfil}\par
\moveleft\hoffset\vbox{\hrule width \resumewidth height 1pt}
\hbox to \textwidth{\hfil 6604 SE 83rd Avenue Portland, Oregon 97266\hfil}
\hbox to \textwidth{\hfil (541) 990-8978\hfil}
\hbox to \textwidth{\hfil beachc@gmail.com\hfil}
\hbox to \textwidth{\hfil github.com/cbeach\hfil}


\begin{resume}

\section{SUMMARY OF QUALIFICATIONS} 
    I have lead several projects and participated in many others. In my personal life, my projects are 
    chosen based on marketability and difficulty. My preference is for massive, difficult projects have 
    a great deal of utility. These most often involve some form of data 
    collection/analysis or cutting edge AI.
    
    I am currently actively pursuing projects in big data, machine learning, and AI.  Below is a short 
    list of my expertise and interests.  

    \begin{itemize}
        \item Data analysis and collection, particularly Twitter's streaming API and BitCoin market data
        \item Researching novel ways to visualize and encode data for easy human consumption
        \item Conceiving and writing solutions that utilize machine learning and artificial intelligence
    \end{itemize}

\section{SKILLS}
    \begin{multicols}{3}
        \textbf{Languages}
        \begin{itemize}
            \item Python
            \item JavaScript
            \item Java
            \item Scala
            \item C/C++
        \end{itemize}
        \columnbreak

        \textbf{Tools}
        \begin{itemize}
            \item Vim
            \item Git
            \item Valgrind
            \item kCacheGrind
            \item Eclipse
            \item Elasticsearch
            \item Logstash
            \item Kibana
            \item AWS (EC2, EMR, S3)
        \end{itemize}
        \columnbreak

        \textbf{Libraries and Frameworks}
        \begin{itemize}
            \item NetflixOSS
            \item Apache Spark
            \item Django
            \item Python Twisted
            \item Python Pandas
            \item Flask \& SQLAlchemy
        \end{itemize}
    \end{multicols}

\section{EXPERIENCE}
\vspace{0.125in}

\begin{itemize}[leftmargin=0in]
    \item[] 
        \textbf{Nike Dreams: Senior Application Engineer} \hfill \textbf{September 2014-Present} \\[-0.1in] \rule{\textwidth}{0.5pt}
        The Dreams project aims to bring near-real time monitoring and analytics to Nike. It heavily utilizes the NetflixOSS stack for managing
        AWS and several apache tools for stream processing. Finally, it uses both Couchbase and the ELK (Elasticsearch, Logstash, Kibana) 
        stack for storage, search, and visualization.
        \vspace{0.125in}
        \item[] 
            \begin{samepage}
                \textbf{Dream Walker} \hfill \textbf{September 2014-Present} \\
                Dream Walker is a Java application designed to replay data into the Dreams pipeline. It is used for
                data recovery and replay.
                \begin{itemize}
                    \item[\textbullet] Allows users to recover data from a specific date range
                    \item[\textbullet] Allows users to replay interesting data
                    \item[\textbullet] Data is fed into the system in a way consistent with normal data processing
                \end{itemize}
            \end{samepage}
            \vspace{0.125in}
        \item[] 
            \begin{samepage}
                \textbf{Elasticsearch, Logstash, Kibana} \hfill \textbf{September 2014-Present} \\
                We are using the ELK stack for search and visualization. This allows business users to explore and understand the data which is being collected.
                \begin{itemize}
                    \item[\textbullet] Authored a Kibana 4 sidecar which gives the Kibana server access to the NetflixOSS stack
                    \item[\textbullet] Allowed business users to visuallize near real time data by deploying and managing an ELK stack
                    \item[\textbullet] Elasticsearch and Kibana has allowed business users to search and visualize traffic to Nike.com in near real time
                    \item[\textbullet] Facilitate rapid debugging by forwarding all Spark cluster system logs to ElasticSearch
                \end{itemize}
            \end{samepage}
\end{itemize}
\vspace{0.25in}

\begin{itemize}[leftmargin=0in]
    \item[] 
        \textbf{PARTHENON SOFTWARE: Lead Software Developer} \hfill \textbf{December 2013-August 2014} \\[-0.1in] \rule{\textwidth}{0.5pt}
        Parthenon specializes in custom software. I have completed projects on a wide variety of technologies including basic 
        web development, telecommunications software, order fulfillment, and others. \vspace{0.125in}
        \begin{itemize}[leftmargin=0in]
            \item[] 
                \begin{samepage}
                    \textbf{Feedback Innovations} \hfill \textbf{December 2013-August 2014} \\
                    Feedback Innovations is a company that collects customer satisfaction data for medical transport. Their original site 
                    used PHP and was not extensible.  An updated version was written in Python and used another sub-contractor Parthenon 
                    was hired to salvage the second version.
                    \begin{itemize}
                        \item[\textbullet] Advised Jon Hance (project manager) and Joe Stepp (developer) on refactoring and estimates
                        \item[\textbullet] Optimized report generation. Large reports no longer crash the server
                        \item[\textbullet] Completely re-wrote the CSV user import tool. It is more powerful, stable, and extensible, with
                                           less than half the code
                    \end{itemize}
                \end{samepage}
        \end{itemize}
\end{itemize}
\vspace{0.25in}

\begin{itemize}[leftmargin=0in]
    \item[] \textbf{Artificial Inteligence, Machine Learning, Genetic Programming, and Computer Vision} \\[-0.1in] \rule{\textwidth}{0.5pt}
        \begin{itemize}[leftmargin=0in]
            \item[] 
                \begin{samepage}
                    \textbf{Deep Thought} \hfill \textbf{May 2012-Present} \\
                    This work in progress is an attempt to implement the Atari playing, deep learning AI from Deep Mind. Deep Thought uses computer vision to isolate game artifacts and 
                    sprites from the NES. The goal is to vastly reduce the size of the input layer that a naive implementation of DeepMind would require. 
                    This will improve both the speed of learning, and the quality of the final neural network.
                    \begin{itemize}
                        \item[\textbullet] Analyzes NES games using computer vision and machine learning
                        \item[\textbullet] Determines the size of repeating tiles by using fast fourier transforms
                        \item[\textbullet] Extract sprites for player and non-player characters (PC, NPC respecitvely)
                        \item[\textbullet] Statistically analyze input and output to find correlation between input and on-screen behavior
                        \item[\textbullet] Perform optical character recognition to extract textual information from individual frames
                    \end{itemize}
                \end{samepage}
            \item[] 
                \begin{samepage}
                    \textbf{LeisurelyScript Game Description Language} \hfill \textbf{May 2012-Present} \\
                    A game description language (GDL) is a computer language that describes the rules of a game. 
                    This project was inspired by my interest in machine learning and AI. It has been spurred on by the 
                    Stanford general game playing MOOC. The ultimate goal is to explore the properties that make a game
                    fun for use in evaluation of new games produced using genetic programming techniques. 
                    The LeisurelyScript language draws heavily from the language 
                    outlined in Cameron Browne's "Automatic Generation and Evaluation".
                    \begin{itemize}
                        \item[\textbullet] LeisurelyScript uses Flex and Bison to generate the scanner and parser
                        \item[\textbullet] The compiler is used for the generation of a state machine expressed in C++
                        \item[\textbullet] The C++ state machine is then compiled into dll's using g++
                        \item[\textbullet] The state machine implements a standard API that an AI can query for required information
                    \end{itemize}
                \end{samepage}
        \end{itemize}
\end{itemize}
\vspace{0.25in}

\end{resume}
\end{document}




