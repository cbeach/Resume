%LaTeX resume using res.cls
\documentclass[margin]{res}
%\usepackage{helvetica} % uses helvetica postscript font (download helvetica.sty)
%\usepackage{newcent}   % uses new century schoolbook postscript font 
\setlength{\textwidth}{5.1in} % set width of text portion

\usepackage{multicol}
\usepackage{latexsym}

\begin{document}
\pagestyle{empty}
\topmargin 0pt
\leftmargin -.5in

% Center the name over the entire width of resumef:
 \moveleft.5\hoffset\centerline{\large\bf Casey M. Beach}
% Draw a horizontal line the whole width of resume:
 \moveleft\hoffset\vbox{\hrule width\resumewidth height 1pt}\smallskip
% address begins here
% Again, the address lines must be centered over entire width of resume:
 \moveleft.5\hoffset\centerline{3263 SE 178th Avenue}
 \moveleft.5\hoffset\centerline{Portland, OR 97236}
 \moveleft.5\hoffset\centerline{(541) 990-8978}
 \moveleft.5\hoffset\centerline{beachc@gmail.com}
 \moveleft.5\hoffset\centerline{github.com/cbeach}
\begin{resume}

\section{EDUCATION} {\sl Current Student,} Computer Science \\
                Portland State University, Portland, OR. \\
                Expected Graduation Date: March 2012 \\
                {\sl Bachcelor of Science,} Business Administration \\
                Oregon State University, Corvallis, OR, 
 				Graduation Date: March 2007 \\
 				Concentration: Entrepreneurship
 
\section{OBJECTIVE} A position in a small company that pushes my abilities in computer science to their 
					limits, and forces me to be the best software engineer possible.

\section{EXPERIENCE}
				
				{\sl \textbf{Computer Science Tutor}} \hfill \textbf{2011-Present} \\
                College of Computer Science, Portland State University \hfill
				
				Mentor freshmen, sophomore, and junior students.  Teaching them basic and intermediate 
				computer science principles and techniques.	I've hosted several introductory workshops 
				on using Linux, and basic Python.
				
				{\sl \textbf{Aerial Vehicles Team}}\hfill \textbf{2012-Present} \\
				College of Computer Science, Portland State University

				I joined the Aerial Vehicle Team at PSU to participate in a cutting edge robotics
				competition.  I will be working on the visual systems that will help the AVT's quad-
				copter navigate.  This system will have several parts.  The first will be a machine
				learning program that performs photo opticle character recognition.  The second will
				be a vision system that will identify objects that need to be avoided.  Finally it 
				will require navigation software that will allow it to identify and avoid obsticles
				while flying quickly through a hallway.

				{\sl \textbf{Twitter Streaming API Cache Project Lead}} \hfill \textbf{2012-Present} \\
				College of Computer Science, Portland State University \\
				Github.com/cbeach/twitter\_vamp 

				I am working on a server that will passively collect public status updates from the
				Twitter streaming api.  These status updates will be stored so that other PSU student
				and myself may do natural langauge processing, semantic analysis, and datamining on
				a very large dataset.				

				{\sl \textbf{3D Printing and Advanced Fabrication}} \hfill \textbf{2010-2012} \\
                Open Tech-Lab, Portland State University
				
				I have completed an open source RepRap 3D printer. The printer started as a solo project, 
				but since the completion of the initial build a team has been found to assist with further 
				upgrades and research.  I built the RepRap from scratch.  I fabricated the structure, soldered  
				the electronics, and flashed the firmware. 	The goals of this project are to advance the 
				fields of robotic manufacturing and replication, to assist other projects at PSU, and to 
				provide a teaching platform for students to learn more about rapid prototyping. 
				More information can be found at http://projects.cecs.pdx.edu/projects/opentech-reprap. 
	

\end{resume}
\end{document}




