%LaTeX resume using res.cls
\documentclass[overlapped]{res}

%\usepackage{helvetica} % uses helvetica postscript font (download helvetica.sty)
%\usepackage{newcent}   % uses new century schoolbook postscript font 

\usepackage{multicol}
\usepackage{latexsym}
\usepackage{enumitem}
\setlist[1]{noitemsep}

\begin{document}
\pagestyle{empty}

\leavevmode\hbox to \textwidth{\hfil\namefont Casey M. Beach\hfil}\par
\moveleft\hoffset\vbox{\hrule width \resumewidth height 1pt}
\hbox to \textwidth{\hfil 6604 SE 83rd Avenue Portland, Oregon 97266\hfil}
%\hbox to \textwidth{\hfil (541) 990-8978\hfil}
\hbox to \textwidth{\hfil beachc@gmail.com\hfil}
\hbox to \textwidth{\hfil github.com/cbeach\hfil}


\begin{resume}

% META: begin.section.skills
\section{SKILLS}
    \begin{multicols}{4}
        \textbf{Languages}
        \begin{itemize}
            \item Python
            \item Ansible
            \item Node/JavaScript
            \item Java
            \item C/C++
            \item Scala
        \end{itemize}
        \columnbreak

        \textbf{AWS}
        \begin{itemize}
            \item EC2 
            \item ECS 
            \item Lambda 
            \item EMR 
            \item S3 
            \item DynamoDB
            \item Route53
            \item SNS
            \item CodeCommit
            \item CodeDeploy
        \end{itemize}
        \columnbreak

        \textbf{Tools}
        \begin{itemize}
            \item Vim
            \item Git
            \item Valgrind
            \item kCacheGrind
            \item Docker
            \item ELK Stack
            \item LinkerD
            \item NamerD
        \end{itemize}
        \columnbreak

        \textbf{Frameworks}
        \begin{itemize}
            \item Node.js
            \item NetflixOSS
            \item Apache Spark
            \item Django
            \item Flask \& SQLAlchemy
        \end{itemize}
    \end{multicols}
% META: end.section.skills

% META: begin.section.experience
\section{EXPERIENCE}
\vspace{0.125in}

% META: begin.section.experience.cambia
\begin{itemize}[leftmargin=0in]
    \item[] 
        \textbf{Cambia Health Solutions: Senior Application Engineer III} \hfill \textbf{February 2016-Present} \\[-0.1in] \rule{\textwidth}{0.5pt}
        Cambia is currently working on a migration to the cloud. As part of the CloudFirst initiative I have had an opportunity to add value to all aspects of 
        the migration from on-prem to AWS based infrastructure.
        \vspace{0.125in}
        % META: begin.section.experience.cambia.triggered
        \item[] 
            \begin{samepage}
                \textbf{Triggered Campaigns} \hfill \textbf{June 2017-Present} \\
                Triggered campaigns give the marketing department a first class interface into a high capacity email pipeline.
                \begin{itemize}
                    \item[\textbullet] Integrated logging and error pipeline using an ELK stack
                    \item[\textbullet] EMR jobs are triggered by PUT events on an S3 bucket
                    \item[\textbullet] A series of node based dockerized microservices then handles validating messages
                    \item[\textbullet] Integration of elastic beats agents to collect logs and system metrics
                \end{itemize}
            \end{samepage}
            \vspace{0.125in}
        % META: end.section.experience.cambia.triggered
        % META: begin.section.experience.cambia.ci_cd_pipeline
        \item[] 
            \begin{samepage}
                \textbf{CI/CD Pipeline} \hfill \textbf{October 2016-Present} \\
                I designed our CI/CD pipeline to give my team a stable baseline for my our HIPPA compliant docker based infrastructure.
                \begin{itemize}
                    \item[\textbullet] Ansible IAC for baking and deploying ec2 instances
                    \item[\textbullet] First team in Cambia to deploy dockerized services to production
                    \item[\textbullet] All microservices are run in docker containers
                    \item[\textbullet] Gitlab Ci for building containers and NPM modules
                    \item[\textbullet] A linkerD POC service mesh is currently being investigated
                \end{itemize}
            \end{samepage}
            \vspace{0.125in}
        % META: end.section.experience.cambia.ci_cd_pipeline
        % META: begin.section.experience.cambia.ci_cd_cps
        \item[] 
            \begin{samepage}
                \textbf{Consumer Profile Settings} \hfill \textbf{February 2016-March 2016} \\
                Consumer profile settings (CPS) stores member information in a HIPAA compliant manner.
                \begin{itemize}
                    \item[\textbullet] Stores HIPAA PII and PHI in searchable encrypted dynamoDB tables
                    \item[\textbullet] NodeJS based services are built and deployed as docker containers
                    \item[\textbullet] Containers are built using gitlab-ci, and then deployed via Jenkins
                    \item[\textbullet] Uses ansible to provision resources in AWS
                    \item[\textbullet] Developed several npm modules for use as utilities
                \end{itemize}
            \end{samepage}
            \vspace{0.125in}
        % META: end.section.experience.cambia.cps
\end{itemize}
\vspace{0.25in}
% META: end.section.experience.cambia

% META: begin.section.experience.Nike
\begin{itemize}[leftmargin=0in]
    \item[] 
        \textbf{Nike Dreams: Senior Application Engineer} \hfill \textbf{September 2014-June 2015} \\[-0.1in] \rule{\textwidth}{0.5pt}
        The Dreams project aims to bring near-real time monitoring and analytics to Nike. It heavily utilizes the NetflixOSS stack for managing
        AWS and several apache tools for stream processing. Finally, it uses both Couchbase and the ELK (Elasticsearch, Logstash, Kibana) 
        stack for storage, search, and visualization.
        \vspace{0.125in}
        \begin{itemize}[leftmargin=0in]
            \item[] 
                % META: begin.section.experience.nike.dream_walker
                \begin{samepage}
                    \textbf{Dream Walker} \hfill \textbf{September 2014-June 2015} \\
                    Dream Walker is a Java application designed to replay data into the Dreams pipeline. It is used for
                    data recovery and replay.
                    \begin{itemize}
                        \item[\textbullet] Allows users to recover data from a specific date range
                        \item[\textbullet] Allows users to replay interesting data
                        \item[\textbullet] Data is fed into the system in a way consistent with normal data processing
                    \end{itemize}
                \end{samepage}
                % META: end.section.experience.nike.dream_walker
                % META: begin.section.experience.nike.elk
                \begin{samepage}
                    \textbf{Elasticsearch, Logstash, Kibana} \hfill \textbf{September 2014-June 2015} \\
                    We are using the ELK stack for search and visualization. This allows business users to explore and understand the data which is being collected.
                    \begin{itemize}
                        \item[\textbullet] Authored a Kibana 4 sidecar which gives the Kibana server access to the NetflixOSS stack
                        \item[\textbullet] Allowed business users to visuallize near real time data by deploying and managing an ELK stack
                        \item[\textbullet] Elasticsearch and Kibana has allowed business users to search and visualize traffic to Nike.com in near real time
                        \item[\textbullet] Facilitate rapid debugging by forwarding all Spark cluster system logs to ElasticSearch
                    \end{itemize}
                \end{samepage}
                % META: end.section.experience.nike.elk
        \end{itemize}
\end{itemize}
\vspace{0.25in}
% META: begin.section.experience.Nike

% META: begin.section.experience.apind
\begin{itemize}[leftmargin=0in]
    \item[] \textbf{Accident Prone Industries} \\[-0.1in] \rule{\textwidth}{0.5pt}
    Accident Prone Industries is a banner name for several related projects that I have been working on for several years.
        \begin{itemize}[leftmargin=0in]
            % META: begin.section.experience.apind.gaas
            \item[] 
                \begin{samepage}
                    \textbf{Gaming as a Service: GaaS} \hfill \textbf{May 2012-Present} \\
                    Gaming as a Service provides a media server experience for friends and family. The emulators are instrumented such that 
                    video output and controller state are saved for future analysis.
                    \begin{itemize}
                        \item[\textbullet] Instrumented emulation for use by friends and family.
                        \item[\textbullet] Uses gRPC for highly portable and documented api's
                    \end{itemize}
                \end{samepage}
            % META: end.section.experience.apind.gaas
            % META: begin.section.experience.apind.deep_thought
            \item[] 
                \begin{samepage}
                    \textbf{Deep Thought} \hfill \textbf{May 2012-Present} \\
                    This work is an attempt to implement computer vision tools to isolate game artifacts and 
                    sprites from the NES using Apache Spark, and finally to extend the Deep Mind reinforcement Neural network to the NES. 
                    The goal is to vastly reduce the size of the input layer that a naive implementation of DeepMind would require. 
                    This will improve both the speed of learning, and the quality of the final neural network.
                    \begin{itemize}
                        \item[\textbullet] Automatically extract sprites from live play sessions
                        \item[\textbullet] Analyzes NES games using computer vision and machine learning
                        \item[\textbullet] Statistically analyze input and output to find correlation between input and on-screen behavior
                    \end{itemize}
                \end{samepage}
            % META: end.section.experience.apind.deep_thought
            % META: begin.section.experience.apind.leisurely_script
            \item[] 
                \begin{samepage}
                    \textbf{LeisurelyScript Game Description Language} \hfill \textbf{May 2012-Present} \\
                    A game description language (GDL) is a computer language that describes the rules of a game. 
                    This project was inspired by my interest in machine learning and AI. It has been spurred on by the 
                    Stanford general game playing MOOC. The ultimate goal is to explore the properties that make a game
                    fun for use in evaluation of new games produced using genetic programming techniques. 
                    The LeisurelyScript language draws heavily from the language 
                    outlined in Cameron Browne's "Automatic Generation and Evaluation".
                    \begin{itemize}
                        \item[\textbullet] LeisurelyScript uses Flex and Bison to generate the scanner and parser
                        \item[\textbullet] The compiler is used for the generation of a state machine expressed in C++
                        \item[\textbullet] The C++ state machine is then compiled into dll's using g++
                        \item[\textbullet] The state machine implements a standard API that an AI can query for required information
                    \end{itemize}
                \end{samepage}
            % META: end.section.experience.apind.leisurely_script
        \end{itemize}
\end{itemize}
% META: end.section.experience.apind


\end{resume}
\end{document}
