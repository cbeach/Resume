%LaTeX resume using res.cls
\documentclass[overlapped]{res}

%\usepackage{helvetica} % uses helvetica postscript font (download helvetica.sty)
%\usepackage{newcent}   % uses new century schoolbook postscript font 

\usepackage{multicol}
\usepackage{latexsym}
\usepackage{enumitem}
\setlist[1]{noitemsep}

\begin{document}
\pagestyle{empty}

\leavevmode\hbox to \textwidth{\hfil\namefont Casey M. Beach\hfil}\par
\moveleft\hoffset\vbox{\hrule width \resumewidth height 1pt}
\hbox to \textwidth{\hfil 6604 SE 83rd Avenue Portland, Oregon 97266\hfil}
\hbox to \textwidth{\hfil (541) 990-8978\hfil}
\hbox to \textwidth{\hfil beachc@gmail.com\hfil}
\hbox to \textwidth{\hfil github.com/cbeach\hfil}


\begin{resume}

\section{SUMMARY OF QUALIFICATIONS} 
    I have lead several projects and participated in many others. In my personal life, my projects are 
    chosen based on marketability and difficulty. My preference is for massive, difficult projects that 
    have a great deal of utility and marketability. These most often involve some form of data 
    collection/analysis or cutting edge AI.
    
    My professional experience has included a wide range of technologies and solutions.  These include
    e-commerce, data analysis, telecommunication, and more. I was recently given the title of lead developer and 
    the opportunity to mentor a recent college graduate.
    
    I am currently actively pursuing projects in big data, machine learning, and AI.  Below is a short 
    list of my expertise and interests.  

    \begin{itemize}
        \item Data analysis and collection, particularly Twitter's streaming API and BitCoin market data
        \item Researching novel ways to visualize and encode data for easy human consumption
        \item Conceiving and writing solutions that utilize machine learning and artificial intelligence
    \end{itemize}

\section{SKILLS}
    \begin{multicols}{3}
        \textbf{Languages}
        \begin{itemize}
            \item Python
            \item JavaScript
            \item Java
            \item C/C++
            \item PHP
        \end{itemize}
        \columnbreak

        \textbf{Tools}
        \begin{itemize}
            \item Vim
            \item Git
            \item Valgrind
            \item kCacheGrind
            \item Eclipse
        \end{itemize}
        \columnbreak

        \textbf{Libraries and Frameworks}
        \begin{itemize}
            \item Flex/Bison
            \item Storm
            \item Python Pandas
            \item Django
            \item Python Twisted
            \item Google Analytics
        \end{itemize}
    \end{multicols}

\section{EXPERIENCE}
\vspace{0.125in}

\begin{itemize}[leftmargin=0in]
    \item[] \textbf{PERSONAL PROJECTS} \\[-0.1in] \rule{\textwidth}{0.5pt}
        \begin{itemize}[leftmargin=0in]
            \item[] 
                \begin{samepage}
                    \textbf{Bitcoin/Twitter Data Collection} \hfill \textbf{December 2013-February 2014} \\
                    This project collects streams of data from Twitter and several bitcoin exchanges with Storm. Storm is an open 
                    source framework for "Distributed and fault-tolerant real-time computation", that is language agnostic,
                    guarantees delivery, and is easily distributed over clusters of any size. This project's purpose was to gain experience
                    with data collection and analysis tools. 
                    \begin{itemize}
                        \item[\textbullet] Storm uses a graph like abstraction for representing the flow of data through various analyses
                        \item[\textbullet] Nodes in Storm's graph represent an algorithm, directed edges represent data flow between algoritms
                        \item[\textbullet] Storm is language agnostic, nodes in this project were written in Python due to familiarity.
                        \item[\textbullet] Collection code was written for Twitter, Mt. Gox, Btc-e, CoinBase, and other exchanges
                        \item[\textbullet] A graph analysis tool was written for arbitrage analysis.
                    \end{itemize}
                \end{samepage}
            \item[] 
                \begin{samepage}
                    \textbf{LeisurelyScript Game Description Language} \hfill \textbf{May 2012-Present} \\
                    A game description language (GDL) is a computer language that describes the rules of games. 
                    This project was inspired by an interest in machine learning and AI, and has been spurred on by the 
                    Stanford general game playing MOOC. The ultimate goal is to explore the properties that make a game
                    fun, and to use those properties for the evaluation of new games produced using genetic programming techniques. 
                    The LeisurelyScript language draws heavily from the language outlined in Cameron Browne's 
                    "Automatic Generation and Evaluation".
                    \begin{itemize}
                        \item[\textbullet] LeisurelyScript uses Flex and Bison to generate the scanner and parser
                        \item[\textbullet] The compiler is used for the generation of a state machine expressed in C++
                        \item[\textbullet] The C++ state machine is then compiled into dll's using g++
                        \item[\textbullet] The state machine implements a standard API that an AI can query for required information
                    \end{itemize}
                \end{samepage}
        \end{itemize}
\end{itemize}
\vspace{0.25in}

\begin{itemize}[leftmargin=0in]
    \item[] \textbf{Nike DREAMS Project: Application Engineer} \\[-0.1in] \rule{\textwidth}{0.5pt}
        \begin{itemize}[leftmargin=0in]
            \item[] 
                \begin{samepage}
                    \textbf{} \hfill \textbf{September 2014-Present 2014} \\
                    The DREAMS project aims to bring real time analytics in house at Nike. It uses a microservice architecture to achieve very fast processing time. It utilizes AWS and the NetflixFloss stack.
                    \begin{itemize} I am owner of DreamWalker, a microservice that rapidly streams data out of S3 and directly into kafka.
                        \item[\textbullet] 
                    \end{itemize}
                \end{samepage}
        \end{itemize}
\end{itemize}
\vspace{0.25in}

\begin{itemize}[leftmargin=0in]
    \item[] 
        \textbf{PARTHENON SOFTWARE: Lead Software Developer} \hfill \textbf{December 2013-Present} \\[-0.1in] \rule{\textwidth}{0.5pt}
        Parthenon specializes in custom software. I have completed projects on a wide variety of technologies including basic 
        web development, telecommunications software, order fulfillment, and others. \vspace{0.125in}
        \begin{itemize}[leftmargin=0in]
            \item[] 
                \begin{samepage}
                    \textbf{Feedback Innovations} \hfill \textbf{December 2013-Present} \\
                    Feedback Innovations is a company that collects customer satisfaction data for medical transport. Their original site 
                    used PHP and was not extensible.  An updated version was written in Python and used another sub-contractor Parthenon 
                    was hired to salvage the second version.
                    \begin{itemize}
                        \item[\textbullet] Advised Jon Hance (project manager) and Joe Stepp (developer) on refactoring and estimates
                        \item[\textbullet] Optimized report generation. Large reports no longer crash the server.
                        \item[\textbullet] Completely re-wrote the CSV user import tool. It is more powerful, stable, and extensible, with
                                           less than half the code.
                    \end{itemize}
                \end{samepage}
        \end{itemize}
        \vspace{0.125in}
    \item[] 
        \textbf{PARTHENON SOFTWARE: Software Developer} \hfill \textbf{April 2012-Present} \\[-0.1in] \rule{\textwidth}{0.5pt}
        \begin{itemize}[leftmargin=0in]
            \item[] 
                \begin{samepage}
                    \textbf{Avaya Bridge Conferencing} \hfill \textbf{July 2012-December 2013} \\
                    Monitoring and control software for Spectel teleconferencing bridges. I was the sole developer for much of the early project.
                    \begin{itemize}
                        \item[\textbullet] Utilizes Python Twisted, Django 1.4, Bridge communication API, web sockets, non-blocking IO
                        \item[\textbullet] Heavily concurrent. Uses concurrent event driven code to perform tasks.
                        \item[\textbullet] Most communication happens over web-sockets and uses Python Twisted to serve them.
                        \item[\textbullet] Developed an RPC framework for twisted web sockets that works well with the Django ORM
                        \item[\textbullet] An event driven Java server handles communication over multiple bridges
                        \item[\textbullet] Only other available software was a Java application written in the late '90's
                        \item[\textbullet] Clients realized a large improvement over the previous version of software
                        \item[\textbullet] Can now handle $>$ 2000 caller conferences which would easily crash the previous version.
                    \end{itemize}
                \end{samepage}
        \end{itemize}
\end{itemize}
\vspace{0.25in}

\section{EDUCATION} 
\begin{multicols}{2}
    \textbf{Bachelor's Degree, Computer Science} \\
    Portland State University, Portland Oregon. \\
    Graduation Date: March 2012 \\

    \columnbreak
    \textbf{Bachcelor's Degree , Business Admin.} \\
    Oregon State University, Corvallis Oregon, \\
    Graduation Date: March 2007 \\
    Concentration: Entrepreneurship
\end{multicols}


\end{resume}
\end{document}




