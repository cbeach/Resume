%LaTeX resume using res.cls
\documentclass[overlapped]{res}

%\usepackage{helvetica} % uses helvetica postscript font (download helvetica.sty)
%\usepackage{newcent}   % uses new century schoolbook postscript font 

\usepackage{multicol}
\usepackage{latexsym}
\usepackage{enumitem}
\setlist[1]{noitemsep}

\begin{document}
\pagestyle{empty}

\leavevmode\hbox to \textwidth{\hfil\namefont Casey M. Beach\hfil}\par
\moveleft\hoffset\vbox{\hrule width \resumewidth height 1pt}
\hbox to \textwidth{\hfil 6604 SE 83rd Avenue Portland, Oregon 97266\hfil}
\hbox to \textwidth{\hfil (541) 990-8978\hfil}
\hbox to \textwidth{\hfil beachc@gmail.com\hfil}
\hbox to \textwidth{\hfil github.com/cbeach\hfil}


\begin{resume}

\section{SUMMARY OF QUALIFICATIONS} 
    I have lead several projects and participated in many others. In my personal life, my projects are 
    chosen based on marketability and difficulty. My preference is for massive, difficult projects have 
    a great deal of utility. These most often involve some form of data 
    collection/analysis or cutting edge AI.
    
    I am currently actively pursuing projects in big data, machine learning, and AI.  Below is a short 
    list of my expertise and interests.  

    \begin{itemize}
        \item Data analysis and collection, particularly Twitter's streaming API and BitCoin market data
        \item Researching novel ways to visualize and encode data for easy human consumption
        \item Conceiving and writing solutions that utilize machine learning and artificial intelligence
    \end{itemize}

\section{SKILLS}
    \begin{multicols}{3}
        \textbf{Languages}
        \begin{itemize}
            \item Python
            \item Node/JavaScript
            \item Java
            \item C/C++
            \item Scala
        \end{itemize}
        \columnbreak

        \textbf{Tools}
        \begin{itemize}
            \item Vim
            \item Git
            \item Valgrind
            \item kCacheGrind
            \item AWS (EC2, ECS, Lambda, EMR, S3, DynamoDB)
            \item Docker
            \item Rancher
            \item ELK Stack
        \end{itemize}
        \columnbreak

        \textbf{Libraries and Frameworks}
        \begin{itemize}
            \item NetflixOSS
            \item Apache Spark
            \item Django
            \item Python Twisted
            \item Python Pandas
            \item Node.js
            \item Flask \& SQLAlchemy
        \end{itemize}
    \end{multicols}

\section{EXPERIENCE}
\vspace{0.125in}

\begin{itemize}[leftmargin=0in]
    \item[] 
        \textbf{Cambia Health Solutions: Senior Application Engineer III} \hfill \textbf{February 2016-Present} \\[-0.1in] \rule{\textwidth}{0.5pt}
        Cambia stuff blah blah blah
        \vspace{0.125in}
        \item[] 
            \begin{samepage}
                \textbf{Consumer Profile Settings} \hfill \textbf{February 2016-Present} \\
                CPS crap
                \begin{itemize}
                    \item[\textbullet] Ansible!
                    \item[\textbullet] Docker!
                    \item[\textbullet] Rancher!
                    \item[\textbullet] AWS!
                    \item[\textbullet] NodeJS
                \end{itemize}
            \end{samepage}
            \vspace{0.125in}
        \item[] 
            \begin{samepage}
                \textbf{Operation Walled Garden} \hfill \textbf{October 2016-Present} \\
                Describe MCB
                \begin{itemize}
                    \item[\textbullet] EMR
                    \item[\textbullet] DynamoDB
                    \item[\textbullet] NodeJS
                \end{itemize}
            \end{samepage}
            \vspace{0.125in}
        \item[] 
            \begin{samepage}
                \textbf{Marketing Campaign Builder} \hfill \textbf{December 2016-Present} \\
                Describe MCB
                \begin{itemize}
                    \item[\textbullet] EMR
                    \item[\textbullet] DynamoDB
                    \item[\textbullet] NodeJS
                \end{itemize}
            \end{samepage}
            \vspace{0.125in}
        \item[] 
            \begin{samepage}
                \textbf{Messaging Service} \hfill \textbf{January 2017-Present} \\
                Describe MCB
                \begin{itemize}
                    \item[\textbullet] EMR
                    \item[\textbullet] DynamoDB
                    \item[\textbullet] NodeJS
                \end{itemize}
            \end{samepage}
\end{itemize}
\vspace{0.25in}

\begin{itemize}[leftmargin=0in]
    \item[] 
        \textbf{Nike Dreams: Senior Application Engineer} \hfill \textbf{September 2014-Present} \\[-0.1in] \rule{\textwidth}{0.5pt}
        The Dreams project aims to bring near-real time monitoring and analytics to Nike. It heavily utilizes the NetflixOSS stack for managing
        AWS and several apache tools for stream processing. Finally, it uses both Couchbase and the ELK (Elasticsearch, Logstash, Kibana) 
        stack for storage, search, and visualization.
        \vspace{0.125in}
        \begin{itemize}[leftmargin=0in]
            \item[] 
                \begin{samepage}
                    \textbf{Dream Walker} \hfill \textbf{September 2014-Present} \\
                    Dream Walker is a Java application designed to replay data into the Dreams pipeline. It is used for
                    data recovery and replay.
                    \begin{itemize}
                        \item[\textbullet] Allows users to recover data from a specific date range
                        \item[\textbullet] Allows users to replay interesting data
                        \item[\textbullet] Data is fed into the system in a way consistent with normal data processing
                    \end{itemize}
                \end{samepage}
                \begin{samepage}
                    \textbf{Elasticsearch, Logstash, Kibana} \hfill \textbf{September 2014-Present} \\
                    We are using the ELK stack for search and visualization. This allows business users to explore and understand the data which is being collected.
                    \begin{itemize}
                        \item[\textbullet] Authored a Kibana 4 sidecar which gives the Kibana server access to the NetflixOSS stack
                        \item[\textbullet] Allowed business users to visuallize near real time data by deploying and managing an ELK stack
                        \item[\textbullet] Elasticsearch and Kibana has allowed business users to search and visualize traffic to Nike.com in near real time
                        \item[\textbullet] Facilitate rapid debugging by forwarding all Spark cluster system logs to ElasticSearch
                    \end{itemize}
                \end{samepage}
        \end{itemize}
\end{itemize}
\vspace{0.25in}


\begin{itemize}[leftmargin=0in]
    \item[] 
        \textbf{PARTHENON SOFTWARE: Lead Software Developer} \hfill \textbf{December 2013-August 2014} \\[-0.1in] \rule{\textwidth}{0.5pt}
        Parthenon specializes in custom software. I have completed projects on a wide variety of technologies including basic 
        web development, telecommunications software, order fulfillment, and others. \vspace{0.125in}
        \begin{itemize}[leftmargin=0in]
            \item[] 
                \begin{samepage}
                    \textbf{Feedback Innovations} \hfill \textbf{December 2013-August 2014} \\
                    Feedback Innovations is a company that collects customer satisfaction data for medical transport. Their original site 
                    used PHP and was not extensible.  An updated version was written in Python and used another sub-contractor Parthenon 
                    was hired to salvage the second version.
                    \begin{itemize}
                        \item[\textbullet] Advised Jon Hance (project manager) and Joe Stepp (developer) on refactoring and estimates
                        \item[\textbullet] Optimized report generation. Large reports no longer crash the server
                        \item[\textbullet] Completely re-wrote the CSV user import tool. It is more powerful, stable, and extensible, with
                                           less than half the code
                    \end{itemize}
                \end{samepage}
                \begin{samepage}
                    \textbf{2nDealer} \hfill \textbf{December 2013- April 2014} \\
                    2nDealer is an online service that offers inventory management for businesses that deal with a large quantity of unique items. 
                    Parthenon was contracted to add new features and stability.
                    \begin{itemize}
                        \item[\textbullet] Makes consignment inventory tracking more efficient
                        \item[\textbullet] Fixed a multitude of bugs
                        \item[\textbullet] Added many features including:
                            \begin{itemize} 
                                \item[\textbullet] A front end interface for label printers 
                                \item[\textbullet] Financial reporting
                                \item[\textbullet] Customer import via CSV 
                            \end{itemize}
                    \end{itemize}
                \end{samepage}
        \end{itemize}
        \vspace{0.125in}
    \item[] 
        \textbf{PARTHENON SOFTWARE: Software Developer} \hfill \textbf{April 2012-August 2014} \\[-0.1in] \rule{\textwidth}{0.5pt}
        \begin{itemize}[leftmargin=0in]
            \item[] 
                \begin{samepage}
                    \textbf{Avaya Bridge Conferencing} \hfill \textbf{July 2012-December 2013} \\
                    Monitoring and control software for Spectel teleconferencing bridges. I was the sole developer for much of the early project.
                    \begin{itemize}
                        \item[\textbullet] Utilizes Python Twisted, Django 1.4, bridge communication API, web sockets, non-blocking IO
                        \item[\textbullet] Heavily concurrent. Uses concurrent event driven code to perform tasks
                        \item[\textbullet] Most communication happens over web-sockets and uses Python Twisted to serve them
                        \item[\textbullet] I developed a RPC framework for twisted web sockets that works well with the Django ORM
                        \item[\textbullet] An event driven Java server handles communication over multiple bridges
                        \item[\textbullet] Only other available software was a Java application written in the late '90's
                        \item[\textbullet] Clients realized a large improvement over the previous version of software
                        \item[\textbullet] Can now handle more than 2000 caller conferences which would easily crash the previous version
                    \end{itemize}
                \end{samepage}
            \item[]
                \begin{samepage}
                    \textbf{Precision Dynamics International} \hfill \textbf{May 2012- September 2012} \\
                    Contractor portal used to coordinate and invoice contracted drivers.
                    \begin{itemize}
                        \item[\textbullet] Utilizes Python 2.7, Django 1.4, Latex, Python-Excel, among others
                        \item[\textbullet] Contractors can sign on, create a profile, and create invoices that managers later approve
                        \item[\textbullet] Managers at PDI can search for contractors using many different criteria
                        \item[\textbullet] Contractor profiles and experience can be exported to a resume, or to an Excel spreadsheet
                        \item[\textbullet] First major client
                    \end{itemize}
                \end{samepage}
        \end{itemize}
\end{itemize}
\vspace{0.25in}

\begin{itemize}[leftmargin=0in]
    \item[] \textbf{Artificial Inteligence, Machine Learning, Genetic Programming, and Computer Vision} \\[-0.1in] \rule{\textwidth}{0.5pt}
        \begin{itemize}[leftmargin=0in]
            \item[] 
                \begin{samepage}
                    \textbf{Deep Thought} \hfill \textbf{May 2012-Present} \\
                    This work in progress is an attempt to implement the Atari playing, deep learning AI from Deep Mind. Deep Thought uses computer vision to isolate game artifacts and 
                    sprites from the NES. The goal is to vastly reduce the size of the input layer that a naive implementation of DeepMind would require. 
                    This will improve both the speed of learning, and the quality of the final neural network.
                    \begin{itemize}
                        \item[\textbullet] Analyzes NES games using computer vision and machine learning
                        \item[\textbullet] Determines the size of repeating tiles by using fast fourier transforms
                        \item[\textbullet] Extract sprites for player and non-player characters (PC, NPC respecitvely)
                        \item[\textbullet] Statistically analyze input and output to find correlation between input and on-screen behavior
                        \item[\textbullet] Perform optical character recognition to extract textual information from individual frames
                    \end{itemize}
                \end{samepage}
            \item[] 
                \begin{samepage}
                    \textbf{LeisurelyScript Game Description Language} \hfill \textbf{May 2012-Present} \\
                    A game description language (GDL) is a computer language that describes the rules of a game. 
                    This project was inspired by my interest in machine learning and AI. It has been spurred on by the 
                    Stanford general game playing MOOC. The ultimate goal is to explore the properties that make a game
                    fun for use in evaluation of new games produced using genetic programming techniques. 
                    The LeisurelyScript language draws heavily from the language 
                    outlined in Cameron Browne's "Automatic Generation and Evaluation".
                    \begin{itemize}
                        \item[\textbullet] LeisurelyScript uses Flex and Bison to generate the scanner and parser
                        \item[\textbullet] The compiler is used for the generation of a state machine expressed in C++
                        \item[\textbullet] The C++ state machine is then compiled into dll's using g++
                        \item[\textbullet] The state machine implements a standard API that an AI can query for required information
                    \end{itemize}
                \end{samepage}
\end{itemize}


\begin{itemize}[leftmargin=0in]
    \item[] \textbf{PERSONAL PROJECTS} \\[-0.1in] \rule{\textwidth}{0.5pt}
            \item[] 
                \begin{samepage}
                    \textbf{Bitcoin/Twitter Data Collection} \hfill \textbf{December 2013-February 2014} \\
                    This project utilizes Storm to collect streams of data from several bitcoin exchanges. This 
                    project's purpose was to gain experience with data collection and analysis tools. Utilizes 
                    Storm; an open source framework for "Distributed and fault-tolerant real-time computation"
                    \begin{itemize}
                        \item[\textbullet] Storm uses a graph like abstraction for representing the flow of data through various analyses
                        \item[\textbullet] Each node in Storm's graph is an algorithm, each directed edge represents input/output data flows
                        \item[\textbullet] Storm is language agnostic, nodes were written in Python due familiarity
                        \item[\textbullet] Collection code was written for Twitter, Mt. Gox, Btc-e, CoinBase, etc.
                        \item[\textbullet] A graph analysis tool was written for arbitrage analysis
                    \end{itemize}
                \end{samepage}
            \item[] 
                \begin{samepage}
                    \textbf{Minor Projects} \hfill \textbf{December 2013-February 2014} \\
                    I have several very small ongoing project in addtion to those listed above.
                    \begin{itemize}
                        \item[\textbullet] Computer controlled kegerator:  A kegerator that posts temperature readings to a database
                        \item[\textbullet] Sous Vide:  A method of slow cooking that uses a water bath set to a very precise temperature
                        \item[\textbullet] Walking Desk:  A Nordictrack c2300 treadmill converted into a desk. 200 miles and counting
                    \end{itemize}
                \end{samepage}
        \end{itemize}
\end{itemize}
\vspace{0.25in}

\begin{itemize}[leftmargin=0in]
    \item[] \textbf{Portland State University} \\[-0.1in] \rule{\textwidth}{0.5pt}
        \begin{itemize}[leftmargin=0in]
            \item[] 
                \begin{samepage}
                    \textbf{Computer Science Tutor} \hfill \textbf{2011-March 2012} \\
                    College of Computer Science, Portland State University \hfill

                    Mentor freshmen, sophomore, and junior students. Help them grasp basic to intermediate level
                    principles and techniques of computer science.	
                    \begin{itemize}
                        \item[\textbullet] Hosted several introductory workshops on Linux and basic Python
                    \end{itemize}
                \end{samepage}

            \item[] 
                \begin{samepage}
                    \textbf{Supportland.com Capstone} \hfill \textbf{2011-March 2021} \\
                    College of Computer Science, Portland State University \hfill

                    Myself and my fellow team mates were responsible for writing a wordpress plugin 
                    for supportland.com. My responsibilities included system administration, as 
                    well as coding. During the duration of this project I have learned a great
                    deal about PHP, javascript, and web development as a whole.
                \end{samepage}

            \item[] 
                \begin{samepage}

                    \textbf{Twitter Streaming API Cache Project Lead} \hfill \textbf{2012-Present} \\
                    College of Computer Science, Portland State University \\
                    Github.com/cbeach/twitter\_vamp 

                    I am working on a server that will passively collect public status updates from the
                    Twitter streaming api. These status updates will be stored so that other PSU student
                    and myself may do natural langauge processing, semantic analysis, and datamining on
                    a very large dataset.				
                \end{samepage}

            \item[] 
                \begin{samepage}

                    \textbf{3D Printing and Advanced Fabrication} \hfill \textbf{2010-2012} \\
                    Open Tech-Lab, Portland State University

                    I have completed an open source RepRap 3D printer. The printer started as a solo project, 
                    but since the completion of the initial build a team has been found to assist with further 
                    upgrades and research. I built the RepRap from scratch. I fabricated the structure, soldered  
                    the electronics, and flashed the firmware. 	The goals of this project are to advance the 
                    fields of robotic manufacturing and replication, to assist other projects at PSU, and to 
                    provide a teaching platform for students to learn more about rapid prototyping. 
                    More information can be found at http://projects.cecs.pdx.edu/projects/opentech-reprap. 
                \end{samepage}
        \end{itemize}
\end{itemize}
	
\section{EDUCATION} 
\begin{multicols}{2}
    \textbf{Bachelor's Degree, Computer Science} \\
    Portland State University, Portland Oregon. \\
    Graduation Date: March 2012 \\

    \columnbreak
    \textbf{Bachcelor's Degree , Business Admin.} \\
    Oregon State University, Corvallis Oregon, \\
    Graduation Date: March 2007 \\
    Concentration: Entrepreneurship
\end{multicols}


\end{resume}
\end{document}




