%LaTeX resume using res.cls
\documentclass[overlapped]{res}

%\usepackage{helvetica} % uses helvetica postscript font (download helvetica.sty)
%\usepackage{newcent}   % uses new century schoolbook postscript font 

\usepackage{multicol}
\usepackage{latexsym}
\usepackage{enumitem}
\setlist[1]{noitemsep}

\begin{document}
\pagestyle{empty}

\leavevmode\hbox to \textwidth{\hfil\namefont Casey M. Beach\hfil}\par
\moveleft\hoffset\vbox{\hrule width \resumewidth height 1pt}
\hbox to \textwidth{\hfil 6604 SE 83rd Avenue Portland, Oregon 97266\hfil}
\hbox to \textwidth{\hfil (541) 990-8978\hfil}
\hbox to \textwidth{\hfil beachc@gmail.com\hfil}
\hbox to \textwidth{\hfil github.com/cbeach\hfil}


\begin{resume}

\section{SUMMARY OF QUALIFICATIONS} 
    I have lead several projects and participated in many others. In my personal life, my projects are 
    chosen based on marketability and difficulty. My preference is for massive, difficult projects have 
    a great deal of utility and or marketability. These most often involve some form of data 
    collection/analysis or cutting edge AI.
    
    My professional experience has included a wide range of technologies and solutions.  These range from
    telecommunication, e-commerce, and data analysis. I was recently given the title of lead developer and 
    given the opportunity to mentor a recent college graduate.
    
    I am currently actively pursuing projects in big data, machine learning, and AI.  Below is a short 
    list of my expertise and interests.  

    \begin{itemize}
        \item Data analysis and collection, particularly Twitter's streaming API and BitCoin market data
        \item Researching novel ways to visualize and encode data for easy human consumption
        \item Conceiving and writing solutions that utilize machine learning and artificial intelligence
    \end{itemize}

\section{SKILLS}
    \begin{multicols}{3}
        \textbf{Languages}
        \begin{itemize}
            \item Python
            \item JavaScript
            \item Java
            \item C/C++
            \item PHP
        \end{itemize}
        \columnbreak

        \textbf{Tools}
        \begin{itemize}
            \item Vim
            \item Git
            \item Valgrind
            \item kCacheGrind
            \item Eclipse
        \end{itemize}
        \columnbreak

        \textbf{Libraries and Frameworks}
        \begin{itemize}
            \item Django
            \item Flask \& SQLAlchemy
            \item Backbone.js
            \item Google Analytics
            \item Storm
            \item Flex/Bison
        \end{itemize}
    \end{multicols}

\section{EXPERIENCE}
\vspace{0.125in}
\begin{itemize}[leftmargin=0in]
    \item[] \textbf{PERSONAL PROJECTS} \\[-0.1in] \rule{\textwidth}{0.5pt}
        \begin{itemize}[leftmargin=0in]
            \item[] 
                \begin{samepage}
                    \textbf{Bitcoin/Twitter Data Collection} \hfill \textbf{December 2013-February 2014} \\
                    This project utilizes Storm to collect streams of data from several bitcoin exchanges. This 
                    project's purpose was to gain experience with data collection and analysis tools. Utilizes 
                    Storm; an open source framework for "Distributed and fault-tolerant real-time computation"
                    \begin{itemize}
                        \item[\textbullet] Storm uses a graph like abstraction for representing the flow of data through various analyses
                        \item[\textbullet] Each node in Storm's graph is an algorithm, each directed edge represents input/output data flows
                        \item[\textbullet] Storm is language agnostic, nodes were written in Python due familiarity.
                        \item[\textbullet] Collection code was written for Twitter, Mt. Gox, Btc-e, CoinBase, etc.
                        \item[\textbullet] A graph analysis tool was written for arbitrage analysis.
                    \end{itemize}
                \end{samepage}
            \item[] 
                \begin{samepage}
                    \textbf{LeisurelyScript Game Description Language} \hfill \textbf{May 2012-Present} \\
                    A game description language (GDL) is a computer language that describes the rules of a game. 
                    This project was inspired by my interest in machine learning and AI. It has been spurred on by the 
                    Stanford general game playing MOOC. The ultimate goal is to explore the properties that make a game
                    fun for use in evaluation of new games produced using genetic programming techniques. 
                    The LeisurelyScript language draws heavily from the language 
                    outlined in Cameron Browne's "Automatic Generation and Evaluation".
                    \begin{itemize}
                        \item[\textbullet] LeisurelyScript uses Flex and Bison to generate the scanner and parser
                        \item[\textbullet] The compiler is used for the generation of a state machine expressed in C++
                        \item[\textbullet] The C++ state machine is then compiled into dll's using g++
                        \item[\textbullet] The state machine implements a standard API that an AI can query for required information
                    \end{itemize}
                \end{samepage}
            \item[] 
                \begin{samepage}
                    \textbf{Minor Projects} \hfill \textbf{December 2013-February 2014} \\
                    I have several very small ongoing project in addtion to those listed above.
                    \begin{itemize}
                        \item[\textbullet] Computer controlled kegerator:  A kegerator that posts temperature readings to a database.
                        \item[\textbullet] Sous Vide:  A method of slow cooking that uses a water bath set to a very precise temperature
                        \item[\textbullet] Walking Desk:  A Nordictrack c2300 treadmill converted into a desk. 200 miles and counting.
                    \end{itemize}
                \end{samepage}
        \end{itemize}
\end{itemize}
\vspace{0.25in}

\begin{itemize}[leftmargin=0in]
    \item[] 
        \textbf{PARTHENON SOFTWARE: Lead Software Developer} \hfill \textbf{December 2013-Present} \\[-0.1in] \rule{\textwidth}{0.5pt}
        Parthenon specializes in custom software. I have completed projects on a wide variety of technologies including basic 
        web development, telecommunications software, order fulfillment, and others. \vspace{0.125in}
        \begin{itemize}[leftmargin=0in]
            \item[] 
                \begin{samepage}
                    \textbf{Feedback Innovations} \hfill \textbf{December 2013-Present} \\
                    Feedback Innovations is a company that collects customer satisfaction data after an ambulance ride. Their original site 
                    was written in such a way that it was not extensible. Another developer was hired through a third party to re-write the 
                    entire site. The original developer disappeared, leaving the site in a very poor state.
                    \begin{itemize}
                        \item[\textbullet] Advised Jon Hance (project manager) and Joe Stepp (developer) on refactoring and estimates
                        \item[\textbullet] Optimized report generation. Large reports no longer crashes the server.
                        \item[\textbullet] Completely re-wrote the CSV user import tool
                    \end{itemize}
                \end{samepage}

                \begin{samepage}
                    \textbf{2nDealer} \hfill \textbf{December 2013- April 2014} \\
                    2nDealer is an online service that offers inventory management for business that deal with a large quantity of unique items. 
                    The original developers were a pair of freelancers that put a flat rate bid and could not keep up with substantial feature creep.
                    \begin{itemize}
                        \item[\textbullet] Originally developed for medical consignment services.
                        \item[\textbullet] Fixed a multitude of bugs
                        \item[\textbullet] Added features including financial accounting reports, and CSV customer import
                    \end{itemize}
                \end{samepage}
        \end{itemize}
        \vspace{0.125in}

    \pagebreak
    \item[] 
        \textbf{PARTHENON SOFTWARE: Software Developer} \hfill \textbf{April 2012-Present} \\[-0.1in] \rule{\textwidth}{0.5pt}
        \begin{itemize}[leftmargin=0in]
            \item[] 
                \begin{samepage}
                    \textbf{Avaya Bridge Conferencing} \hfill \textbf{July 2012-December 2013} \\
                    Monitoring and control software for Spectel teleconferencing bridges. I was the sole developer for much of the early project.
                    \begin{itemize}
                        \item[\textbullet] Utilizes Python Twisted, Django 1.4, Bridge communication API, web sockets, non-blocking IO
                        \item[\textbullet] Heavily concurrent. Uses concurrent event driven code to perform tasks.
                        \item[\textbullet] Most communication happens over web-sockets and uses Python Twisted to serve them.
                        \item[\textbullet] I developed a RPC framework for twisted web sockets that works well with the Django ORM
                        \item[\textbullet] An event driven Java server handles communication over multiple bridges
                        \item[\textbullet] Only other available software was a Java application written in the late '90's
                        \item[\textbullet] Clients realized a large improvement over the previous version of software
                        \item[\textbullet] Can now $>$ 2000 caller conferences which would easily crash the previous version.
                    \end{itemize}
                \end{samepage}
            \item[]
                \begin{samepage}
                    \textbf{Precision Dynamics International} \hfill \textbf{May 2012- September 2012} \\
                    Contractor portal used to coordinate and invoice contracted drivers.
                    \begin{itemize}
                        \item[\textbullet] Utilizes Python 2.7, Django 1.4, Latex, Python-Excel, among others.
                        \item[\textbullet] Contractors can sign on, create a profile, and create invoices that managers later approve.
                        \item[\textbullet] Managers at PDI can search for contractors using many different criteria.
                        \item[\textbullet] Contractor profiles and experience can be exported to a resume, or to an Excel spreadsheet.
                        \item[\textbullet] First major client.
                    \end{itemize}
                \end{samepage}
        \end{itemize}
\end{itemize}

\vspace{0.25in}

\begin{itemize}[leftmargin=0in]
    \item[] \textbf{Portland State University} \\[-0.1in] \rule{\textwidth}{0.5pt}
        \begin{itemize}[leftmargin=0in]
            \item[] 
                \begin{samepage}
                    \textbf{Computer Science Tutor} \hfill \textbf{2011-Present} \\
                    College of Computer Science, Portland State University \hfill

                    Mentor freshmen, sophomore, and junior students. Help them grasp basic to intermediate level
                    principles and techniques of computer science.	
                    \begin{itemize}
                        \item[\textbullet] Hosted several introductory workshops on Linux and basic Python.
                    \end{itemize}
                \end{samepage}

            \item[] 
                \begin{samepage}
                    \textbf{Supportland.com} \hfill \textbf{2011-Present} \\
                    College of Computer Science Capstone, Portland State University \hfill

                    Myself and my fellow team mates were responsible for writing a wordpress plugin 
                    for supportland.com. My responsibilities included system administration, as 
                    well as coding. During the duration of this project I have learned a great
                    deal about PHP, javascript, and web development as a whole.
                \end{samepage}

            \item[] 
                \begin{samepage}

                    \textbf{Twitter Streaming API Cache Project Lead} \hfill \textbf{2012-Present} \\
                    College of Computer Science, Portland State University \\
                    Github.com/cbeach/twitter\_vamp 

                    I am working on a server that will passively collect public status updates from the
                    Twitter streaming api. These status updates will be stored so that other PSU student
                    and myself may do natural langauge processing, semantic analysis, and datamining on
                    a very large dataset.				
                \end{samepage}

            \item[] 
                \begin{samepage}

                    \textbf{3D Printing and Advanced Fabrication} \hfill \textbf{2010-2012} \\
                    Open Tech-Lab, Portland State University

                    I have completed an open source RepRap 3D printer. The printer started as a solo project, 
                    but since the completion of the initial build a team has been found to assist with further 
                    upgrades and research. I built the RepRap from scratch. I fabricated the structure, soldered  
                    the electronics, and flashed the firmware. 	The goals of this project are to advance the 
                    fields of robotic manufacturing and replication, to assist other projects at PSU, and to 
                    provide a teaching platform for students to learn more about rapid prototyping. 
                    More information can be found at http://projects.cecs.pdx.edu/projects/opentech-reprap. 
                \end{samepage}
        \end{itemize}
\end{itemize}
	
\section{EDUCATION} 
\begin{multicols}{2}
    \textbf{Bachelor's Degree, Computer Science} \\
    Portland State University, Portland, OR. \\
    Graduation Date: March 2012 \\
    \columnbreak
    \textbf{Bachcelor of Science, Business Administration} \\
    Oregon State University, Corvallis, OR, 
    Graduation Date: March 2007 \\
    Concentration: Entrepreneurship
\end{multicols}


\end{resume}
\end{document}




