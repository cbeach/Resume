%LaTeX resume using res.cls
\documentclass[overlapped]{res}

%\usepackage{helvetica} % uses helvetica postscript font (download helvetica.sty)
%\usepackage{newcent}   % uses new century schoolbook postscript font 

\usepackage{multicol}
\usepackage{latexsym}
\usepackage{enumitem}
\setlist[1]{noitemsep}

\begin{document}
\pagestyle{empty}

\leavevmode\hbox to \textwidth{\hfil\namefont Casey M. Beach\hfil}\par
\moveleft\hoffset\vbox{\hrule width \resumewidth height 1pt}
\hbox to \textwidth{\hfil 11956 SE 34th Avenue Portland, Oregon 97219\hfil}
%\hbox to \textwidth{\hfil (541) 990-8978\hfil}
\hbox to \textwidth{\hfil beachc@gmail.com\hfil}
\hbox to \textwidth{\hfil github.com/cbeach\hfil}

\begin{resume}
\section{SKILLS}
    \begin{multicols}{3}
        \textbf{Languages}
        \begin{itemize}
            \item Python
            \item Go
            \item Node/JavaScript
            \item Java
            \item C/C++
            \item Scala
        \end{itemize}
        \columnbreak

        \textbf{Tools}
        \begin{itemize}
            \item Vim
            \item Git
            \item Valgrind
            \item kCacheGrind
            \item AWS (EC2, ECS, Lambda, EMR, S3, DynamoDB)
            \item Docker
            \item Rancher
            \item ELK Stack
        \end{itemize}
        \columnbreak

        \textbf{Libraries and Frameworks}
        \begin{itemize}
            \item Go echo
            \item Apache Spark
            \item Django
            \item Python Pandas
            \item NodeJS
            \item Flask \& SQLAlchemy
        \end{itemize}
    \end{multicols}

\section{EXPERIENCE}
\vspace{0.125in}


\begin{itemize}[leftmargin=0in]
    \item[] 
        \textbf{Gametime United: Senior Developer} \hfill \textbf{March 2019-October 2019} \\[-0.1in] \rule{\textwidth}{0.5pt}
        %TODO: This needs more work
        As part of Gametime's backend developer team I was responsible for feature work as well as maintenance and developer support for our internal tooling.
        As a senior developer I also contributed to architectural discussions and researched/lead efforts to improve our codebase and tooling.
        \vspace{0.125in}
        \item[] 
            \textbf{Mongo Migration} \hfill \textbf{March-August 2019} \\
            When I started at Gametime, plans were underway to migrate from a self managed mongo cluster, to one managed by Atlas.
            The intention being to save developer time and money, while simultaneously increasing quality and reliability.
            \begin{itemize}
                \item[\textbullet] Backups were being taken of the self hosted cluster, but no one had ever actually checked that they worked. I called this out and subsequently validated that they were correct.
                \item[\textbullet] I performed a cross validation of data on the two clusters at the behest of the VP. Validation included several key metrics, as well as a detailed validation of a statistically significant sample of the clusters.
                %TODO: \item[\textbullet] More things here.
            \end{itemize}
        \vspace{0.125in}

        \textbf{Restful API development in Go} \hfill \textbf{March-October 2019} \\
        \begin{itemize}
            \item[\textbullet] All new Restful API features at Gametime were written in Go using the echo web framework
            \item[\textbullet] Was responsible for converting legacy rails endpoints to Go
            %TODO: \item[\textbullet] More things here.
        \end{itemize}
        \vspace{0.125in}

        \textbf{Bulk refactor and other proposals} \hfill \textbf{June-October 2019} \\
        When first adopted, Go was a new language to Gametime. This led to the codebase developing organically 
        while the code architecture was devised. This organic growth let to maintainability issues. I identified 
        issues such as "type bloat" where many different versions of the same information are present in the same 
        scope and cross service type safety where gRPC/Protbuf was present in the codebase. I worked to resolve these issues by:
        \begin{itemize}
            \item[\textbullet] Devised and worked on a code analysis and refactoring tool inspired by Go Guru.
            \begin{itemize}
              \item[\textbullet] Code refactoring tool would download all Gametime repositories, sort them by language, and then use the go/* packages to parse and analyze the go repos
            \end{itemize}
            \item[\textbullet] Proposed and architected a drop-in solution for the issues with visibility and deployment using Envoy and AWS App Mesh
        \end{itemize}
        \vspace{0.125in}
        % META: end.section.experience.gametime.mongo_migration
\end{itemize}
\vspace{0.25in}

\begin{itemize}[leftmargin=0in]
    \item[] 
        \textbf{Cambia Health Solutions: Senior Application Engineer III} \hfill \textbf{February 2016-March 2019} \\[-0.1in] \rule{\textwidth}{0.5pt}
        %TODO: This needs more work
        As part of Cambia's ongoing migration to the cloud I have had an opportunity to add value to all aspects of our tech stack.
        \vspace{0.125in}
        % META: begin.section.experience.cambia.juno_migration
        \item[] 
            \begin{samepage}
                \textbf{Juno Migration} \hfill \textbf{October 2016-March 2019} \\
                Juno is a new AWS account/environment with better amenities and more "handrails" to help developers deploy their code. 
                A lot of work is required to migrate our tools and services to the new environment. I was chosen to be my team's lead 
                lead in this effort and have been deprecating old, incompatible services in preparation for this effort.
                \begin{itemize}
                    \item[\textbullet] Refactored old foundational libraries/modules to make use of new infrastructure features 
                    \item[\textbullet] Meticulously planned a service mesh integration using Envoy to make use of cutting edge service mesh technology
                    \item[\textbullet] Deprecated 73 very old and incompatible AWS instances and related infrastructure, 
                    \item[\textbullet] The instances alone save Cambia \$20,000 per month
                    \item[\textbullet] Designed a cutting edge service mesh based on envoy and the AWS elastic kubernetes service
                \end{itemize}
            \end{samepage}
            \vspace{0.125in}
        % META: end.section.experience.cambia.juno_migration
        % META: begin.section.experience.cambia.ci_cd_pipeline
        \item[] 
            \begin{samepage}
                \textbf{CI/CD Pipeline} \hfill \textbf{October 2016-March 2019} \\
                I designed our CI/CD pipeline to give my team a stable baseline for my our HIPAA compliant docker based infrastructure.
                \begin{itemize}
                    \item[\textbullet] Designed and wrote Ansible IaC for baking and deploying ec2 instances
                    \item[\textbullet] Helped us become the first team in Cambia to deploy dockerized services to production
                    \item[\textbullet] Designed the system so that all microservices are run in docker containers
                    \item[\textbullet] Gitlab Ci for building containers and NPM modules
                    \item[\textbullet] A linkerD POC service mesh is currently being investigated
                \end{itemize}
            \end{samepage}
            \vspace{0.125in}
        % META: end.section.experience.cambia.ci_cd_pipeline
        % META: begin.section.experience.cambia.ci_cd_cps
        \item[] 
            \begin{samepage}
                \textbf{Consumer Profile Settings} \hfill \textbf{February-March 2016} \\
                I helped design and implement our Consumer profile settings (CPS) project to store member information and preferences in a HIPAA compliant manner.
                \begin{itemize}
                    \item[\textbullet] Stores HIPAA PII and PHI in searchable encrypted dynamoDB tables
                    \item[\textbullet] NodeJS based services are built and deployed as docker containers
                    \item[\textbullet] Containers are built using gitlab-ci, and then deployed via Jenkins
                    \item[\textbullet] Uses Ansible to provision resources in AWS
                    \item[\textbullet] Developed several NPM modules for use as utilities
                \end{itemize}
            \end{samepage}
            \vspace{0.125in}
        % META: end.section.experience.cambia.cps
\end{itemize}
\vspace{0.25in}
\begin{itemize}[leftmargin=0in]
    \item[] 
        \textbf{Nike Dreams: Senior Application Engineer} \hfill \textbf{September 2014-June 2015} \\[-0.1in] \rule{\textwidth}{0.5pt}
        The Dreams project aims to bring near-real time monitoring and analytics to Nike. It heavily utilizes the NetflixOSS stack for managing
        AWS and several Apache tools for stream processing. Finally, it uses both Couchbase and the ELK (Elasticsearch, Logstash, Kibana) 
        stack for storage, search, and visualization.
        \vspace{0.125in}
        \begin{itemize}[leftmargin=0in]
            \item[] 
                \begin{samepage}
                    \textbf{Dream Walker} \hfill \textbf{September 2014-June 2015} \\
                    Dream Walker is a Java application designed to replay data into the Dreams pipeline. It is used for
                    data recovery and replay.
                    \begin{itemize}
                        \item[\textbullet] Designed the application to allow users to recover data from specific date ranges
                        \item[\textbullet] Allows users to replay interesting data
                        \item[\textbullet] Data is fed into the system in a way consistent with normal data processing
                    \end{itemize}
                \end{samepage}
                \begin{samepage}
                    \textbf{Elasticsearch, Logstash, Kibana} \hfill \textbf{September 2014-June 2015} \\
                    We are using the ELK stack for search and visualization. This allows business users to explore and understand the data which is being collected.
                    \begin{itemize}
                        \item[\textbullet] Authored a Kibana 4 sidecar which gives the Kibana server access to the NetflixOSS stack
                        \item[\textbullet] Deployed and managed an ELK stack that allowed business users to visualize near real time data
                        \item[\textbullet] Allowed business users to search and visualize traffic to Nike.com in near real time by utilizing Elasticsearch and Kibana 
                        \item[\textbullet] Facilitated rapid debugging by forwarding all Spark cluster system logs to Elasticsearch
                    \end{itemize}
                \end{samepage}
        \end{itemize}
\end{itemize}
\vspace{0.25in}

\begin{itemize}[leftmargin=0in]
    \item[] \textbf{Accident Prone Industries} \\[-0.1in] \rule{\textwidth}{0.5pt}
        Accident Prone Industries a research project that I have been working on since 2013. It's an effort to expand the work done by Cameron Browne in his paper "Automatic Generation and Evaluation
        of Recombination Games"

        \begin{itemize}[leftmargin=0in]
            \item[] 
                \begin{samepage}
                    \textbf{Deep Thought} \hfill \textbf{May 2012-Present} \\
                    This work is an attempt to implement computer vision tools to isolate game artifacts and 
                    sprites, and game rules from retro games. The ultimate goal is to create an unsupervised tool capable of automatically 
                    reverse engineering games.
                    \begin{itemize}
                        \item[\textbullet] I Analyze NES games using computer vision and machine learning
                        \item[\textbullet] My software determines the size of repeating tiles by using fast fourier transforms
                        \item[\textbullet] I extract sprites for player and non-player characters (PC, NPC respectively)
                        \item[\textbullet] Statistically analyze input and output to find correlation between input and on-screen behavior
                        \item[\textbullet] I perform optical character recognition to extract textual information from individual frames
                    \end{itemize}
                \end{samepage}
            \item[] 
                \begin{samepage}
                    \textbf{Segmentum Solar} \hfill \textbf{April 2018-Present} \\
                    This project aims to provide enough horsepower for my other ML projects. The list of hardware and service inventory that I've assembled follows.
                    \begin{itemize}
                        \item[\textbullet] A 40 core Dell r810 with 256 GB of RAM
                        \item[\textbullet] A home built Intel i9 with 48 GB of RAM and a GTX 1080 TI
                        \item[\textbullet] A 8 core Dell r410 intended for use as a virtual router/gateway.
                        \item[\textbullet] A Synology DS718+ 8TB NAS
                        \item[\textbullet] A bare metal deployment of Kubernetes with the helm package manager.
                    \end{itemize}
                \end{samepage}
            \item[] 
                \begin{samepage}
                    \textbf{LeisurelyScript Game Description Language} \hfill \textbf{May 2012-Present} \\
                    A game description language (GDL) is a computer language that describes the rules of a game. 
                    This project was inspired by my interest in machine learning and AI. It has been spurred on by the 
                    Stanford general game playing MOOC. The ultimate goal is to explore the properties that make a game
                    fun, which can be used for the evaluation of new games produced using genetic programming techniques. 
                    The LeisurelyScript language draws heavily from the language 
                    outlined in Cameron Browne's "Automatic Generation and Evaluation".
                    \begin{itemize}
                        \item[\textbullet] LeisurelyScript uses Scala compiler macros to parse my extension to the language 
                        \item[\textbullet] The compiler is used for the generation of a state machine expressed in Scala 
                        \item[\textbullet] The state machine implements a standard API that an AI can query for required information
                        \item[\textbullet] The state machine will then be fed into a WIP evaluation and genetic programming framework
                        \item[\textbullet] My end goal is to produce a ML ecosystem that is capable of randomly generating game rules

                    \end{itemize}
                \end{samepage}
        \end{itemize}
\end{itemize}
\vspace{0.25in}


\begin{itemize}[leftmargin=0in]
    \item[] 
        \textbf{PARTHENON SOFTWARE: Lead Software Developer} \hfill \textbf{December 2013-August 2014} \\[-0.1in] \rule{\textwidth}{0.5pt}
        Parthenon specializes in custom software. I have completed projects on a wide variety of technologies including basic 
        web development, telecommunications software, order fulfillment, and others. \vspace{0.125in}
        \begin{itemize}[leftmargin=0in]
            \item[] 
                \begin{samepage}
                    \textbf{Feedback Innovations} \hfill \textbf{December 2013-August 2014} \\
                    Feedback Innovations is a company that collects customer satisfaction data for medical transport. Their original site 
                    used PHP and was not extensible.  An updated version was written in Python and used another sub-contractor Parthenon 
                    was hired to salvage the second version.
                    \begin{itemize}
                        \item[\textbullet] Advised Jon Hance (project manager) and Joe Stepp (developer) on refactoring and estimates
                        \item[\textbullet] Optimized report generation. Large reports no longer crash the server
                        \item[\textbullet] Completely re-wrote the CSV user import tool. It is more powerful, stable, and extensible, with
                                           less than half the code
                    \end{itemize}
                \end{samepage}
                \begin{samepage}
                    \textbf{2nDealer} \hfill \textbf{December 2013- April 2014} \\
                    2nDealer is an online service that offers inventory management for businesses that deal with a large quantity of unique items. 
                    Parthenon was contracted to add new features and stability.
                    \begin{itemize}
                        \item[\textbullet] Makes consignment inventory tracking more efficient
                        \item[\textbullet] Fixed a multitude of bugs
                        \item[\textbullet] Added many features including:
                            \begin{itemize} 
                                \item[\textbullet] A front end interface for label printers 
                                \item[\textbullet] Financial reporting
                                \item[\textbullet] Customer import via CSV 
                            \end{itemize}
                    \end{itemize}
                \end{samepage}
        \end{itemize}
        \vspace{0.125in}
    \item[] 
        \textbf{PARTHENON SOFTWARE: Software Developer} \hfill \textbf{April 2012-August 2014} \\[-0.1in] \rule{\textwidth}{0.5pt}
        \begin{itemize}[leftmargin=0in]
            \item[] 
                \begin{samepage}
                    \textbf{Avaya Bridge Conferencing} \hfill \textbf{July 2012-December 2013} \\
                    Monitoring and control software for Spectel teleconferencing bridges. I was the sole developer for much of the early project.
                    \begin{itemize}
                        \item[\textbullet] Utilizes Python Twisted, Django 1.4, bridge communication API, web sockets, non-blocking IO
                        \item[\textbullet] Heavily concurrent. Uses concurrent event driven code to perform tasks
                        \item[\textbullet] Most communication happens over web-sockets and uses Python Twisted to serve them
                        \item[\textbullet] I developed a RPC framework for twisted web sockets that works well with the Django ORM
                        \item[\textbullet] An event driven Java server handles communication over multiple bridges
                        \item[\textbullet] Only other available software was a Java application written in the late '90's
                        \item[\textbullet] Clients realized a large improvement over the previous version of software
                        \item[\textbullet] Can now handle more than 2000 caller conferences which would easily crash the previous version
                    \end{itemize}
                \end{samepage}
            \item[]
                \begin{samepage}
                    \textbf{Precision Dynamics International} \hfill \textbf{April-September 2012} \\
                    Contractor portal used to coordinate and invoice contracted drivers.
                    \begin{itemize}
                        \item[\textbullet] Utilizes Python 2.7, Django 1.4, Latex, Python-Excel, among others
                        \item[\textbullet] Contractors can sign on, create a profile, and create invoices that managers later approve
                        \item[\textbullet] Managers at PDI can search for contractors using many different criteria
                        \item[\textbullet] Contractor profiles and experience can be exported to a resume, or to an Excel spreadsheet
                        \item[\textbullet] First major client
                    \end{itemize}
                \end{samepage}
        \end{itemize}
\end{itemize}
\vspace{0.25in}



\begin{itemize}[leftmargin=0in]
    \item[] \textbf{PERSONAL PROJECTS} \\[-0.1in] \rule{\textwidth}{0.5pt}
            \item[] 
                \begin{samepage}
                    \textbf{Bitcoin/Twitter Data Collection} \hfill \textbf{December 2013-February 2014} \\
                    This project utilizes Storm to collect streams of data from several bitcoin exchanges. This 
                    project's purpose was to gain experience with data collection and analysis tools. Utilizes 
                    Storm; an open source framework for "Distributed and fault-tolerant real-time computation"
                    \begin{itemize}
                        \item[\textbullet] Storm uses a graph like abstraction for representing the flow of data through various analyses
                        \item[\textbullet] Each node in Storm's graph is an algorithm, each directed edge represents input/output data flows
                        \item[\textbullet] Storm is language agnostic, nodes were written in Python due familiarity
                        \item[\textbullet] Collection code was written for Twitter, Mt. Gox, Btc-e, CoinBase, etc.
                        \item[\textbullet] A graph analysis tool was written for arbitrage analysis
                    \end{itemize}
                \end{samepage}
            \item[] 
                \begin{samepage}
                    \textbf{Minor Projects} \hfill \textbf{December 2013-February 2014} \\
                    I have several very small ongoing project in addition to those listed above.
                    \begin{itemize}
                        \item[\textbullet] Computer controlled kegerator:  A kegerator that posts temperature readings to a database
                        \item[\textbullet] Sous Vide:  A method of slow cooking that uses a water bath set to a very precise temperature
                        \item[\textbullet] Walking Desk:  A Nordictrack c2300 treadmill converted into a desk. 200 miles and counting
                    \end{itemize}
                \end{samepage}
\end{itemize}
\vspace{0.25in}

\begin{itemize}[leftmargin=0in]
    \item[] \textbf{Portland State University} \\[-0.1in] \rule{\textwidth}{0.5pt}
        \begin{itemize}[leftmargin=0in]
            \item[] 
                \begin{samepage}
                    \textbf{Computer Science Tutor} \hfill \textbf{2011-March 2012} \\
                    College of Computer Science, Portland State University \hfill

                    Mentor freshmen, sophomore, and junior students. Help them grasp basic to intermediate level
                    principles and techniques of computer science.	
                    \begin{itemize}
                        \item[\textbullet] Hosted several introductory workshops on Linux and basic Python
                    \end{itemize}
                \end{samepage}

            \item[] 
                \begin{samepage}
                    \textbf{Supportland.com Capstone} \hfill \textbf{2011-March 2021} \\
                    College of Computer Science, Portland State University \hfill

                    Myself and my fellow team mates were responsible for writing a wordpress plugin 
                    for supportland.com. My responsibilities included system administration, as 
                    well as coding. During the duration of this project I have learned a great
                    deal about PHP, javascript, and web development as a whole.
                \end{samepage}

            \item[] 
                \begin{samepage}

                    \textbf{Twitter Streaming API Cache Project Lead} \hfill \textbf{2012-2013} \\
                    College of Computer Science, Portland State University \\
                    Github.com/cbeach/twitter\_vamp 

                    I am working on a server that will passively collect public status updates from the
                    Twitter streaming API. These status updates will be stored so that other PSU student
                    and myself may do natural language processing, semantic analysis, and data-mining on
                    a very large dataset.				
                \end{samepage}

            \item[] 
                \begin{samepage}

                    \textbf{3D Printing and Advanced Fabrication} \hfill \textbf{2010-2012} \\
                    Open Tech-Lab, Portland State University

                    I have completed an open source RepRap 3D printer. The printer started as a solo project, 
                    but since the completion of the initial build a team has been found to assist with further 
                    upgrades and research. I built the RepRap from scratch. I fabricated the structure, soldered  
                    the electronics, and flashed the firmware. 	The goals of this project are to advance the 
                    fields of robotic manufacturing and replication, to assist other projects at PSU, and to 
                    provide a teaching platform for students to learn more about rapid prototyping. 
                    More information can be found at http://projects.cecs.pdx.edu/projects/opentech-reprap. 
                \end{samepage}
        \end{itemize}
\end{itemize}
	
\section{EDUCATION} 
\begin{multicols}{2}
    \textbf{Bachelor's Degree, Computer Science} \\
    Portland State University, Portland Oregon. \\
    Graduation Date: March 2012 \\

    \columnbreak
    \textbf{Bachelor's Degree , Business Admin.} \\
    Oregon State University, Corvallis Oregon, \\
    Graduation Date: March 2007 \\
    Concentration: Entrepreneurship
\end{multicols}


\end{resume}
\end{document}
