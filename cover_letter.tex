% Cover letter using letter.sty
\documentclass{letter} % Uses 10pt
%Use \documentstyle[newcent]{letter} for New Century Schoolbook postscript font
% the following commands control the margins:
\topmargin=-1in    % Make letterhead start about 1 inch from top of page 
\textheight=8in  % text height can be bigger for a longer letter
\oddsidemargin=0pt % leftmargin is 1 inch
\textwidth=6.5in   % textwidth of 6.5in leaves 1 inch for right margin

\begin{document}

\signature{Casey M. Beach}           % name for signature 
\longindentation=0pt                       % needed to get closing flush left
\let\raggedleft\raggedright                % needed to get date flush left
 
 
\begin{letter}{Urban Robotics, Inc. \\
33 NW First Avenue, Suite 200 \\
Portland, OR 97209 }

\begin{center}
{\large\bf Casey M. Beach}
\end{center}
\medskip\hrule height 1pt
\begin{center}
 3263 178th ave,\\
Portland, OR 97236 \\
(541) 990-8978 
\end{center} 
\vfill % forces letterhead to top of page

 
\opening{To whom it may concern} 
\begin{flushleft} 
\noindent I heard about your company through Jen Hanni several months ago, and was excited to see that you were attending the 
PSU engineering career fair.


\noindent I have been interested in robotics for years.  During my exploration of robotics I've focused on 
the software side, but I have learned a great deal about electronics and electro-mechanical systems as well.  
My greatest project to date has been my scratch built RepRap (open source 3D printer), which currently resides in open 
tech space.

\noindent I have many skills that I have developed primarily through the projects that I have done.  My RepRap taught me
a great deal about physical systems and the limitations that they may have.  It has also taught me, through my research into
open source 3D scanning techniques that even the things that seem insurmountable can be easily overcome if you're clever.

\noindent My newest project is a visual system for an autonomous rover that will be used as a test bed for the autonomous vehicle team's (AVT)
quad-rotor.  I will be using the rover to develop several algorithms for autonomous navigation.  The goal of this development
is participation in an autonomous aerial vehicle challenge.  One of the challenges in this competition is to navigate
through a building using signage.  My goal is to develop a simple photo opticle character recognition algorithm to 
compare words and determine a course according to arrows on the signage posted in the halls.  The classes that I have taken,
and the research that I've done has lead me to a system of algorithms that is likely to work, the final hurdle lies in the 
implementation.

\noindent I am very interested in learning more about your company.  
If you would like to know more about me please visit my github repositories at the address below. \\ 

\end{flushleft}
\begin{center}
	\textbf{http://github.com/cbeach}
\end{center}

Feel free to call or email me if you are interested in an interview.

\closing{Sincerely yours,} 
 

 
\encl{resume}  				% Enclosures

\end{letter}
 

\end{document}






