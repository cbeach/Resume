% Cover letter using letter.sty
\documentclass{letter} % Uses 10pt
%Use \documentstyle[newcent]{letter} for New Century Schoolbook postscript font
% the following commands control the margins:
\topmargin=-1in    % Make letterhead start about 1 inch from top of page 
\textheight=8in  % text height can be bigger for a longer letter
\oddsidemargin=0pt % leftmargin is 1 inch
\textwidth=6.5in   % textwidth of 6.5in leaves 1 inch for right margin

\begin{document}

\signature{Casey M. Beach}           % name for signature 
\longindentation=0pt                       % needed to get closing flush left
\let\raggedleft\raggedright                % needed to get date flush left
 
 
\begin{letter}{
    Company Address
}

\begin{center}
{\large\bf Casey M. Beach}
\end{center}
\medskip\hrule height 1pt
\begin{center}
6604 83rd Avenue,\\
Portland, OR 97266 \\
(541) 990-8978 
\end{center} 
\vfill % forces letterhead to top of page

 
\opening{To whom it may concern} 
\begin{flushleft} 

\noindent The first paragraph of your letter should include information on why 
you are writing. Mention the position you are applying for and where you found 
the job listing. Include the name of a mutual contact, if you have one.

\noindent The next section of your cover letter should describe what you have 
to offer the employer. Mention specifically how your qualifications match the 
job you are applying for.

\noindent Conclude your cover letter by thanking the employer for considering 
you for the position. Include information on how you will follow-up, and an
invitation to contact you.

\end{flushleft}
\closing{Sincerely yours,} 
\encl{resume}  				% Enclosures
\end{letter}
\end{document}
